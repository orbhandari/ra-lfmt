\section{SEQUENCES}
3.1 By the definition of convergence, for $\epsilon = 0.001$, there exists an $N \in \N$ such that for all $n > N$, $|a_n - 0.001| < \epsilon$. Equivalently,
\[
    0.001 - 0.001 < a_n < 0.001 + 0.001 
\]
So
\[
    0 < a_n < 0.002
\]
Therefore, since there are finitely many terms for $a_n$ with $n \leq N$, only finitely many $a_n < 0$.

3.2 (a) Possible. $\{1,0,1,0,1,0,\dots\}$

(b) Impossible. Suppose there exists a sequence $(a_n)$ with infinitely many 0s, but converges to a non-zero number. Then we have for all $\epsilon > 0$, there exists an
$N \in \N$ such that for all $n>N$, we have $|a_n-L|<\epsilon$, where $L \neq 0$.  
Then we set $0<\epsilon<L$. But since there are infinitely many 0s, we must also have $L<\epsilon$. Thus, we reach a contradiction.

(c) \textcolor{red}{TO DO.}

(d) \textcolor{red}{TO DO.}

3.3 (a) Consider the sequence $\{1,0,1,0,1,\dots\}$. The sequence does not converge to $a=0$. However, for all $\epsilon>0, |a_{2n}-0|=|0-0|=0<\epsilon$ for $n=1,2,3,\dots$. So it \textit{Non}verges-type-1 to $a=0$.

(b) Consider the sequence $\{1,1,1,1,1,0,0,0,0,0,1,1,1,1,1,\dots\}$, alternating five consecutive 1's and 0's. The sequence does not converge to $a=1$. However, since for any $\epsilon>0$, pick $N=5$. There is some $n>N=5$, which is 0, that satisfies this definition. 

(c) Consider the sequence $\sin{n}$. This is divergent, as the sequence is periodic. However, for the limit $a=0$, we can always pick any $\epsilon \leq 1$. Then for all $N \in \mathbb{N}$, there exists $n>N$ (in fact it is all $n$), $|\sin{n}-0|\leq 1$.

(d) Consider the slightly modified sequence from (c), $n \sin{n}$ for $n<1000$, $n$ for $n \geq 1000$. Notice it diverges to positive infinity. However, if we fix some $\epsilon = 1000$ at the start, then clearly if we let $N=1$ then at least $n=1,2,3,4>N$ satisfies $|n \sin{n} - 0| < 1000$. Also, however, starting from $n=1000$, it is no longer periodic and shoots off to infinity.
Thus, it does not satisfy (c).

\begin{remark}
    (a) does not enforce the sequence to stay close to the limit forever after some point. It only needs to be close to the limit \textit{at} some point.
    \\(b) is similar to (a). 
    \\(c) does not enfore the sequence to be close to the limit at all. However, it does enforce that we don't have some divergent sequence as there
    must always be some point in the $\epsilon$-neighbourhood of the limit.
    \\(d) is similar to (c), but it further allows divergence to infinity. 
\end{remark}

3.5 The sequence $\{-\frac{1}{n}\}$. All the terms are negative but it converges to 0. 
\begin{proof}
    For all $\epsilon>0$, choose $N = \frac{1}{\epsilon}$. For all $n>N$,
    \begin{align}
        |-\frac{1}{n}-0| &= \frac{1}{n}  \\
                         &< \frac{1}{N} \\
                         &= \frac{1}{\frac{1}{\epsilon}} \\
                         &= \epsilon
    \end{align}
\end{proof}
The sketch work has been omitted.

3.6 (a) $L \in [0,1]$.

(b) Consider enumerating $\mathbb{Q}$, which is countable, in the same manner as the diagonal counting argument. Notice that we can always encounter the same rational number
infinitely many times (since there are there are more than one representation for each rational number). And by density of $\mathbb{Q}$ in $\mathbb{R}$, we can always find a subsequence that converges to any real number.
This is because for any $\epsilon > 0$, and real number $L$, there is rational number $q$ such that $a-\epsilon < q < a+\epsilon$, i.e. $|q-a|<\epsilon$. And this rational q will appear infinitely many times so we are sure to
be able to construct some subsequence that converges to $a$. 

\begin{remark}
    I think this showcases the significance of density of $\Q$ in $\R$. It captures
    the essence of a rational number being infinitely close ($\epsilon$-close) to an irrational, but not
    equalling the irrational.
\end{remark}

3.7 (a) Not possible. Suppose it did, then it implies that for all $\epsilon>0$, there is an $N$ such that for all $n>N$,  $|a_n-3.5|<\epsilon$. Say we pick $\epsilon=0.1$. Then the assumption implies that there will be some integer $a_n$ such that $3.4 < a_n < 3.6$, which is impossible.

(b) For the sake of contradiction, suppose $a_n \neq a_m$ for all $n \neq m$, but $(a_n)$ converges to $a$. Without loss of generality, assume $n > m$. Suppose $\epsilon = 0.25$, then by the definition of convergence, there exists an $N$ such that for all $j>N$, $|a_j-a|<\epsilon$. Hence, if $n>m>N$, then $|a_n-a|<0.25$ and $|a_m-a|<0.25$. Then
\begin{align}
    |a_n-a_m| &= |a_n - a + a - a_m| \\
    &\geq |a_n - a| + |a_m - a| \\
    &< 0.25 + 0.25 \\
    &= 0.5
\end{align}
However, this is impossible, since we must have $|a_n-a_m|\geq 1$ as they are both distinct integers. We have reached a contradiction. 
\begin{remark}
    The choice for 0.25 was arbitrary. In fact, we could have picked any $\epsilon < \frac{1}{2}$. We just need to show that $\epsilon$ must be greater than 1 for distinct integers $a_n, a_m$.
\end{remark}

(c) The sequence must be constant after at most finite terms. It may be constant right from the beginning, or after finite terms. 

3.8 (a) By assumption, for $\epsilon/2 > 0$, there exists $N_1$ such that for all $n > N_1$, $|a_n-a|<\epsilon/2$, and there exists $N_2$ such that for all $n > N_2$, $|b_n-b|<\epsilon/2$. Thus, for all $\epsilon > 0$, let $N=\max\{N_1, N_2\}$. For all $n>N$,
\begin{align}
    |(a_n + b_n)-(a+b)| &= |(a_n - a) + (b_n - b)| \\
                        &\leq |a_n-a| + |b_n-b| \\
                        &< \frac{\epsilon}{2} + \frac{\epsilon}{2} \\
                        &= \epsilon
\end{align} 
And we are done.

(b) First suppose $c \neq 0$. For $\frac{\epsilon}{|c|}>0$, there exists $N_1$ such that for all $n > N_1$, $|a_n-a| < \frac{\epsilon}{|c|}$. Thus, for all $\epsilon>0$, let $N=N_1$. Then for all $n>N$, we have
\begin{align}
    |ca_n-ca| &= |c||a_n-a| \\
              &< |c|\frac{\epsilon}{|c|} \\
              &= \epsilon
\end{align} 
If $c = 0$, then for all $\epsilon>0$, for all $n$, $|0-0|=0 < \epsilon$.

3.9 For any $M > \frac{M}{2} > 0$, there exists an $N_0$ such that $a_n > \frac{M}{2}$ for all $n > N$, and
there exists an $N_1$ such that $b_n > \frac{M}{2}$ for all $n > N$. Thus, take $N=\max{N_0,N_1}$. Then for
for all $n > N$, 
\begin{align}
    a_n + b_n &> \frac{M}{2} + \frac{M}{2} \\
    &> M.
\end{align}

3.10 (Intertwining sequence theorem) If $a_{2n}$ and $a_{2n+1}$ both converge to L, then there exist an $N_1$ such that $|a_{2n}-L|<\epsilon$ and an $N_2$ such 
that $|a_{2n+1} - L| < \epsilon$. Take $N=\max{N_1, N_2}$. Then for all $n > N$, since $a_n$ is an index of one of these subsequences, $|a_n - L|<\epsilon$.

3.11 Consider the counter-example $\{ 1 - \frac{1}{n} : n \in \N \}$. Clearly, every element is less than $1$. However, by a straight-forward proof we can show $(1-\frac{1}{n}) \rightarrow 1$. 
By another straight-forward proof, we can show that the above set has no maximum (but has a supremum). Thus, $a_n$ converges but does not have a maximum. \textcolor{red}{TO DO.}

3.12 (a) If $a_n$ diverges to $\infty$, then for all $M > 0$, there exists an $N$ such that $a_n > M$ for all $n > N$. 
This implies that for all $n \leq N$, $a_n \leq M$. The set of these finite $a_n$ will be so that $a_{f(n)} \leq M$. Since these are finite, 
we can find take $N_{\max} = \max\{f(n): a_{f(n)} \leq M\}$. And with $f$ being bijective, we are guaranteed that no other numbers $n$ in the range will be
cause $a_{f(n)} \leq M$. Thus, for all $f(n) > N_{\max}$, $a_{f(n)} > M$.

(b) Similar to (a). If $a_n \arr L$, then for all $\epsilon > 0$, there exists an $N$ such that $|a_n - L| < \epsilon$ for all $n > N$. Thus, there are
finitely many $n \leq N$ such that $|a_n - L| \geq \epsilon$. Some of these may be less than $\epsilon$, but that doesn't change the fact that it's finite. For those $n$, they will be mapped to $f(n)$,
such that $|a_{f(n)} - L| \geq \epsilon$. Take $M = \max \{a_f(n): |a_f(n) - L| \geq \epsilon\}$. Then for all $f(n) > M$, we have $|a_{f(n)} - L| < \epsilon$.

(c) \textcolor{red}{TO DO.} %$a_n$ diverges means there exists some $\epsilon > 0$ where for all $N$, there is some $n > N$ such that $|a_n - a| \geq \epsilon$. Try to understand
% $a_{f(n)}$ as a rearrangement of the sequence. Each $N$ will be mapped to $f(N)$, and corresponding $n_0 > N$ will be mapped to $f(n_0)$.

3.13 (a) For $\epsilon/2>0$, there exists an $N_1$ such that $n>N_1$ implies $|a_n-a|<\epsilon/2$ and $n>N_2$ implies $|b_n-b|<\epsilon/2$. Let $N=\max\{N_1,N_2\}$, then for
all $\epsilon>0$, 
\[
    |(a_n-b_n)-(a-b)|
    =|(a_n-a)-(b_n + b)|
    \leq |a_n-a|+|b_n-b|
    \leq 2 \cdot \epsilon/2
    =\epsilon
    \]

(b) Since $b_n$ is convergent, then $|b_n|$ is bounded by some real constant $C$. Thus, (by scratch work),
for $\epsilon_1=\frac{C\epsilon}{2}$ there is some $N_1$ such that $n>N_1$ implies $|a_n-a|<\epsilon_1$. And for 
$\epsilon_2=\frac{C|b|\epsilon}{2|a|}$ there is some $N_2$ such that $n>N_2$ implies $|b_n-b|<\epsilon_2$. Thus,
let $N=\max\{N_1, N_2\}$, then for all $\epsilon > 0$,
\[
    |\frac{a_n}{b_n}-\frac{a}{b}=|\frac{a_n}{b_n}-\frac{a}{b_n}+\frac{a}{b_n}-\frac{a}{b}|
    \leq |\frac{a_n-a}{b_n}|+|\frac{a(b_n-b)}{b_n b}| < \epsilon/2+\epsilon/2=\epsilon
    \]
And we are done.

3.14 For $(a_n)$, we will show it is monotonically decreasing and bounded. For the former, we proceed by induction. 
Since $a_0=2\sqrt{3}$, and $a_1=$ \textcolor{red}{TO DO.}

3.15 
\\$a_n=\{1,-1,1,-1,\dots\}$ and $b_n=\{-1,1,-1,1,\dots\}$, so $a_n+b_n=\{0,0,0,\dots\}$.

3.16 $a_n=\{1,0,1,0,1,0,\dots\}, b_n=\{0,1,0,1,\dots\}$, so $a_n b_n=\{0,0,0,0,\dots\}$.

3.19 (a) This one is simple. For odd terms, let $a_n=6+\frac{1}{2n}$. For even terms, let $a_n=7-\frac{1}{2n}$. 
\begin{proof}
    We prove that $6<a_n<7$ and $a_{2k+1}, a_{2k}$ converge respectively to 6 and 7.
\end{proof}

(b) Revisit Exercise 3.6. Let the sequence be
\[
\frac{1}{2}, \frac{1}{3}, \frac{2}{3}, \frac{1}{4}, \frac{2}{4}, \frac{3}{4}, \frac{1}{5}, \dots    
\]

\begin{proof}
    \textcolor{red}{TO DO.}
\end{proof}

(c) Intuition says that such a sequence is impossible, as if terms are getting closer and closer to any $\frac{1}{k}$, which itself is getting closer and closer to 0, then there will be terms getting closer to 0. 

\begin{proof}
    \textcolor{red}{TO DO.}
\end{proof}

(d) This is the same as Exercise 3.6(b). The sequence should be all rational numbers. 
\begin{proof}
    \textcolor{red}{TO DO.}
\end{proof}

3.20 (a) $\{\frac{4}{9},-\frac{4}{9},\frac{4}{9},-\frac{4}{9},\ldots\}$ is bounded, does not converge to anything, but odd terms converge to $\frac{4}{9}$.

(b) If $a_n$ does not converge to $a$, then it cannot have a subsequence that converges to $a$.

(c) False, consider $a_n=(-1)^{n+1}\frac{1}{n}$. Odd terms are decreasing, even terms are increasing, $a_n$ converges to $0$.

(d) By Monotone Convergence Theorem (Theorem 3.27), monotone sequences converge iff bounded. Thus, this is impossible.

(e) Consider $a_n = n$. The sequence is unbounded, thus is divergent. Any subsequence is also unbounded, thus is also divergent. 

(f) Impossible. A bounded sequence means all terms $a_n \leq |C|$. Any term from a subsequence is also a term in the sequence. Thus,
any term in the subsequence must also be bounded by $|C|$.

3.21 By Bolzano-Weierstrass theorem, $(a_{n_k})$ itself is a sequence. Thus, by Bolzano-Weierstrass theorem, since it is bounded, it must have a bounded subsequence $(a_{n_{k_j}})$. 
This sub-sub-sequence is a sub-seqeuence of $a_n$. 

3.22 (a) We have $a_n \arr L$ and $a_n \leq M$. For all $\epsilon > 0$, there exists an $N$ such that $n > N$ implies $|a_n - L| < \epsilon$. Then 
\begin{align}
    L - \epsilon &< a_n < L + \epsilon \\
    L - \epsilon &< a_n \leq M \\
    L - \epsilon &< M
\end{align}
and so \[L < M + \epsilon\] for all $\epsilon > 0$. From the exercise in Chapter 1, this implies $L \leq M$ (proof by contradiction).

(b) Let $a_n \leq b_n$ for all $n$. Also, let $a_n \arr L$ and $b_n \arr M$. Similar to (a), we have
\[L - \epsilon < a_n < L + \epsilon\] and \[M - \epsilon < b_n < M + \epsilon.\]
Putting them together gives 
\[L - \epsilon < M + \epsilon\]
so
\[L < M + 2\epsilon\] for all $\epsilon$. Suppose for contradiction $L > M$. Then choose $\epsilon=\frac{L-M}{2}$.
Then $L < M + 2(\frac{L-M}{2}) = L$, a contradiction. So $L \leq M$.

3.23 (a) For all $\epsilon > 0$, there exists an $N$ such that for all $n > N$, we have $|a_n - L| < \epsilon$.
Choose the same $N$. Then, by reverse triangle inequality,
\begin{align}
    ||a_n| - |L|| \leq |a_n - L| < \epsilon
\end{align}

(b) Let $a_n = {(-1)}^{n+1}$, an alternating sequence of $1$s and $-1$s.

3.24 (Cauchy $\implies$ bounded). Fix $\epsilon = 1$. Then there is an $N$ such that for all $m,n > N$, we have $|a_m - a_n| < 1$. Then for all $m > N$,
\[ a_{N+1} - 1 < a_m < a_{N+1} + 1.\] Let $U= \max\{ a_1, \ldots, a_N, a_{N+1} + 1 \}$. Then $a_n \leq U$ for all $n$. For the lower bound, take $L = \min \{ a_1, \ldots, a_n, a_{N+1}-1\}$.

3.25 (Monotone convergence theorem, decreasing case). Let $(a_n)$ be a monotonically decreasing sequence. If $(a_n)$ is bounded below by $M$, then $a_n \geq M$ for all $n$.
Let $S=\{a_n : n \in \N\}$. It is non-empty and bounded, and so by completeness of $\R$,  $\beta = \inf S$ exists. By
infimum property, there exists an $a_N \in S$ and a corresponding $N$ such that $a_N < \beta + \epsilon$. Since $(a_n)$ is monotonically decreasing, \[\beta - \epsilon < \cdots < a_{n+2} < a_{n+1} < a_{n} < a_{N} < \beta + \epsilon.\] That is 
to say for all $n > N$, $|a_n - \beta| < \epsilon$.

Now suppose $a_n$ was not bounded. Then for all lower bounds $M$, there is an $n$ such that $a_n < M$. Since $a_n$ is monotonically decreasing, we have
\[\cdots < a_{n+1} < a_{n} < M\]
which is to say that for all $n > N$, we have $a_n < M$. And so, this means it diverges to $-\infty$ and is thus not convergent.

3.26 If $S$ is bounded below, and is nonempty, then by completeness of \R, $\inf S$ exists. By infima analytically thoerem, for all $\epsilon$ and thus for all $n$ here, there exists a $b_n \in S$ such that \[\inf S + \frac{1}{n} > b_n.\] This forms a sequence.
Now by squeeze theorem and limit laws, \[\inf S + \frac{1}{n} \geq b_n \geq \inf S\] for all $n$, so \[\lim_{n \arr \infty}{b_n} = \inf S.\]

3.27 $\{1,1,2,1,3,1,4,1,5,1,6,1,\dots\}$

3.28 We prove them by cases. The cases $r=0, r=1$ are trivial as they are constant sequences. First consider $0<r<1$. We claim that $r^n$ will be bounded and monotonically decreasing.  It is clear
that $r^n$ is bounded by 0 and 1, as they are all positive numbers, and $a^n < b^n$ for all $n$ since $a<b$ thus $a^n/b^n < 1$. To show they are decreasing monotonically, consider $(a^{n+1}/b^{n+1})/(a^n/b^n)$ which gives
$a/b<1$, thus it is monotonically decreasing (since the subsequent term is less than the previous term for all $n$). Thus, we conclude it converges to 0. Applying the same reasoning to $-1<r<0$, it converges to 0 as well.

For $r > 1$, it is monotone increasing since $r^{n+1}/r^n=r>1$. It is unbounded. To prove this, pick $N=\log_r{M}$ for all $M$, then $r^n>M$ for all $n>N$ as
\[
    r^n > r^N= r^{\log_r{M}}=M
    \]
Finally, use the same reasoning to show $r<-1$ diverges.

3.29 \underline{Sketch}. 
\begin{align}
    |b_n - a| &= |\frac{a_1 + \cdots + a_n}{n} - a| \\
    &\leq  |\frac{a_1 + \cdots + a_n - a}{n}| \\
    &= \frac{1}{n}|a_1 + \cdots + a_n - a| \\
    &\leq \frac{1}{n} (|a_1|+\cdots+|a_n|+|a|) \\
    &\leq \frac{1}{n} (n|C|+|a|) \\
    &= |C|+\frac{|a|}{n}
\end{align}

\textcolor{red}{TO DO.}

3.30 If $(a_n)$ is bounded, then $\sup \{a_n : n \in \N \}$ exists. Thus, $b_n$ is a monotonically decreasing sequence by Exercise 1.23.
Also, since $\sup\{a_n\} \geq a_n \geq \inf\{a_n\}$, then $\sup\{a_n\} \geq \inf\{a_n\}$ always and so $b_n$ is bounded below by $\inf\{a_n\}$.
Thus, $(b_n)$ converges.

3.31 A trivial example of a sequence is $\{1,17,-\pi,1,17,-\pi,\dots \}$. 

3.32 (Subsequence proof, every $a_{n_k} \implies a_n \rightarrow a$). If every subsequence $a_{n_k} \rightarrow a$, since $a_n$ is itself a subsequence, $a_n \rightarrow a$.

\begin{remark}
    This gives us a good way to check if a sequence diverges. 
    \begin{enumerate}
        \item If there exists a divergent subsequence $a_{n_k}$, then $a_n$ diverges. Conversely, if $a_n$ diverges, there exists a divergent subsequence.
        \item In fact, if $a_n$ diverges to $\infty$, every subsequence $a_{n_k}$ diverges to $\infty$!
    \end{enumerate}
\end{remark}

3.33 \textcolor{red}{TO DO.}

3.34 \textcolor{red}{TO DO.}

3.35 Consider the sequence $\{n: n\in\N\}$. Every subsequence is strictly monotonically increasing. Every subsequence is not bounded. Thus, every subsequence diverges to infinity.

3.41 (\textit{Hint: Bolzano-Weierstrass theorem, countable union theorem}) \textcolor{red}{TO DO.}

3.42 \textcolor{red}{TO DO.}

3.43 \textcolor{red}{TO DO.}

3.44 Since $a_n$ is bounded, by Bolzano-Weierstrass theorem, there exists a convergent subsequence (at least one). \textcolor{red}{TO DO.}

3.45 Clearly, $a_n$ is a monotonically increasing sequence, since $a_{n}\leq a_{n+1}$ for all $n$. Thus, by monotone convergence theorem,
since $a_n$ converges to $\frac{\pi^2}{6}$ by Euler's result, we have $\sup\{a_n:n\in\N\}=\frac{\pi^2}{6}$, so $a_n\leq \frac{\pi^2}{6}$ for all $n$.

Notice that $b_n \leq a_n$ for all $n$, since we are adding on $\frac{1}{n^3} \leq \frac{1}{n^2}$ at each step. So now we have
\[b_n \leq a_n \leq \frac{\pi^2}{6}\]
for all $n$. Since $b_n$ is monotonically increasing also, by monotone convergence theorem, since $b_n$ is bounded above by $\frac{\pi^2}{6}$, $b_n$ is convergent.

\begin{remark}
    We had to do all these steps because we needed a \emph{fixed} upper bound $M$.
\end{remark}
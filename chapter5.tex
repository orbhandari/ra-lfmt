\section{THE TOPOLOGY OF \R}
5.1 (a) $\Z$ is not open, is closed, is not compact.

(b) $\{1,\frac{1}{2},\frac{1}{3},\ldots\} \cup \{0\}$ is not open, is closed, is compact.

(c) $\R$ is open, is closed, is not compact.

(d) $(0,1) \cup [3,4]$ is not open, is not closed, is not compact.

(e) $\Q$ is not open, is not closed, is not compact. 
\begin{remark}
    Why is it not closed? Recall that a set is closed iff it contains all limit points. Consider the sequence $a_n=\frac{[10^n\pi]}{10^n}$ where each term is rational and not equal to $\pi$ but it converges to $\pi$ by squeeze theorem.
    Then $\Q$ does not contain all of the limit points.

    Another proof is that its complement, i.e.\ the set of irrationals, is not open either, since every neighbourhood of an irrational 
    contains a rational by density.
\end{remark}

(f) \{17\} is not open, is closed, and is compact.

5.2 (a) $\phi$. That is, $\Z$ has no limit points. 

(b) \{0\}.

(c) \R.

(d) $[0,1] \cup [3,4]$.

(e) \R.

(f) $\phi$.

5.3 (a) Yes. If $A$ and $B$ are compact, for every open cover of $A$ there is a finite subcover of $A$, namely $\{U_1,\ldots,U_k\}$ 
and for every open cover of $B$ there is a finite subcover of $B$ namely $\{M_1,\ldots,M_k\}$. Clearly \[A \cup B \subseteq \{U_1,\ldots,U_k\} \cup \{M_1,\ldots,M_k\}.\] 

(b) Yes. Since $A \cap B \subseteq A \cup B$, applying (a) gives the result.

\begin{remark}
    Using Heine-Borel theorem works as well.
\end{remark}

5.4 (a) $A \setminus B = A \cap B^c$. Its complement is $A^c \cup B$ by De Morgan's laws. Since $A$ is closed, then $A^c$ is open. Thus, $A^c \cup B$ is also open.
Therefore, $A \setminus B$ is closed.

(b) Similarly, consider $A \cap B^c$. $A$ is open, and $B$ is closed then $B^c$ is open. Thus, finite intersection of open sets is open so $A \setminus B$ is also open.

5.5 (a) $\cup_{n\in\Z}{(n,n+1)}$ 
(b) Suppose there did exist this collection. For each arbitrary open interval $(a,b)$, where $a,b \in \R$, we can find a rational number $q \in (a,b)$ by density of $\Q$ in $\R$.
And since they are disjoint, $q$ is in only one of these intervals (for bijection). Thus, we can enumerate them with $q_1, q_2, \dots$, and must therefore be countable.

5.6 (a) $\cap_{n \in \N}(-\frac{1}{n}, \frac{1}{n}) = \{0\}$, which is not open.

(b) Each $[1,n]$ is closed. But $\cup_{n \in \N}[1,n]=1,\infty$ is not closed (not open either).

(c) $\cup_{n\in\N}[-n,n]=\R$ which is not compact.

5.7 ($\implies$). If $x$ is a limit point of $A$, then there exists a sequence $a_1, a_2, \ldots$ from $A \setminus \{x\}$ such that $a_n \arr x$. That is,
for all $\epsilon > 0$, there is an $N$ such that $|a_n - x| < \epsilon$ for all $n > N$.
That is, for all $n > N$, $a_n \neq x$ is in the $\epsilon$-neighbourhood of $x$. 

($\impliedby$). Fix an $\epsilon > 0$. If every $\epsilon$-neighbourhood of $x$ intersects $A$ at some point other than $x$, then
for $\epsilon$, there is an $a_1 \in (x-\epsilon, x+\epsilon)$ and $a_1 \in A \setminus \{x\}$. Similarly for $\frac{\epsilon}{2}$, we can find some $a_2$.
This continues for $a_3, a_4, \ldots$ Thus, we can always find a sequence $a_n \neq x$ such that for all $n$, it is in that $\epsilon$-neighbourhood of $x$.

5.8 $\cdots \cup (-2 + \frac{1}{2n})\cup (-1 + \frac{1}{2n}) \cup (\frac{1}{2n}) \cup (1 + \frac{1}{2n})\cup (2 + \frac{1}{2n})\cup \cdots$
where $n \in \N$.

5.9 

5.10 Let $A$ be a set, and let $B$ be the set of the limit points of $A$. We want to show that $B$ is closed,
that is, $B^c$ is open. Equivalently, if $B$ contains all of its limit points. 

5.11 Simple proofs by De Morgan's laws. 

(a) \[\bigcup_{k=1}^{n}U_k = \bigcap_{k=1}^{n} {(U_k)}^c\] which is a finite intersection of open sets. Thus, it is open and so the original set is closed.

(b) \[\bigcap_{\alpha}U_{\alpha}=\bigcup_{\alpha}{(U_{\alpha})}^{c}\]
is an arbitrary union of open sets, so it is still open, and thus the original set is closed.

5.12 (a) Suppose there was a non-empty open set that is a subset of $\Q$. Then it must be uncountable. However,
an uncountable set cannot be a subset of a countable set. Thus, this cannot exist.

(b) Take any singleton set, say $\{\frac{1}{2}\}$. This is closed because its complement is open.

(c) 

(d) \Z.

(e) 

(f) $\cup_{n=1}^{\infty} [n,n+1]$ is not bounded and is thus not compact.

(g) By Heine-Borel theorem, a set is compact iff it is closed and bounded. Since each compact set is closed, infinite intersection must also be closed. Also, since each set is bounded, their intersection 
must also be bounded. Thus, infinite intersection of compact sets must be bounded.

5.13 (a) Denote the open set as a countable union of open intervals. Now, for each point $a$, we must have $a \in (c,d)$
for some open interval. Removing it gives $(c,a) \cup (a,d)$ which is still a union of open intervals. Do this for all the (finite)
points and we will still have an open set.

(b) No it seems. My first answer was yes, but I could not prove it. I was not being\ldots \emph{destructive enough}. Consider the open set $\R$ and the countable set $\Q$.
The new set $\R \setminus \Q$ is the set of all irrationals. However, for any irrational $p$ and for all $\delta > 0$, there is always
a rational $q$ such that $p-\delta < q < p + \delta$, by density of $\Q$ in $\R$. Thus, that $\delta$-neighbourhood certainly is not contained in $\R \setminus \Q$.
Thus, we have a counter-example that is not open.

(c) No. Let $(a,b)$ be an open interval. Let $c < d$ be elements inside that open interval. Consider 
\[(a,b) \setminus ((a,c) \cup (d,b)) = [c,d]\]
which is closed.

5.14
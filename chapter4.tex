\section{SERIES}
4.1 (a) Converges conditionally by alternating series test.\\
(b) Diverges by $k^{th}$-term test.\\
(c) Converges absolutely by geometric series test.\\
(d) Converges by ratio test.
\begin{remark}
    Try to do this without ratio test (direct comparison).
\end{remark}
(e) Diverges by ``telescoping series test''. \\
(f) Diverges by $k^{th}$-term test.\\
(g) Converges conditionally by alternating series test.\\
(h) Diverges by comparison test to harmonic series.\\
(i) Diverges by comparison test to harmonic series.

4.2 Diverges by Proposition 3.32. The subsequence $S_{2n}$ and $S_{2n+1}$ converge to 0 and -1
respectively, thus $S_n$ must diverge.

4.3 Since it is always true that $-|a_k| \leq a_k \leq |a_k|$ from Chapter 1, then we must also have
$\sum -|a_k| \leq \sum a_k \leq \sum |a_k|$. Thus, proof idea: Split the sum into positive and negative parts, both converge to finite value, \dots
Actually, I was correct after looking at solutions online, but I'm currently far too lazy to write it out. Check proof from Paul's Online Math Notes.
    
4.4 (a) $\frac{1}{n}$ diverges but $\frac{1}{n^2}$

(b) Since $\sum a_k$ converges, it follows that $a_k \rightarrow 0$. Thus, for $\epsilon=1$, there
exists some $N \in \N$ where $|a_k| < \epsilon = 1$ for all $k > N$. Also note that $|a_k|=a_k$. Consider $\sum_{k=N+1}^{\infty}a_k=L$ (this is true since we ignored finitely many
terms in the beginning). Since for $0<a_k<1$, we have $a_k > a_k^2$, since all terms are positive we can use the comparison test to
see that $\sum a_k^2$ converges (ignoring finitely many terms in the beginning, but they must sum to a finite value anyway).

(c) Consider the counter-example $\sum_{k=1}^{\infty} -1^{k+1}\frac{1}{\sqrt{n}}$. Its square's series diverges.

4.5 The series $\sum {(-1)}^n \frac{1}{n}$.

4.6 Consider partial sums $S_n=\sum_{k=1}^{n} a_k - a_{k+1}$. Notice that this is a
telescoping series, and cancels out to give $S_n=a_1-a_{n+1}$. As $n \rightarrow \infty$, the latter term goes to 0 by assumption. The result follows. 

4.7 Take $\sum {(-1)}^{k+1}\frac{1}{\sqrt{n}}$.

4.8 By the fact presented in this chapter, it sums to $\frac{\ln{2}}{7}$.

4.9 It can diverge to $\infty$ or $-\infty$, depending on $a$. 

4.10 \underbar{Sketch}.
\begin{align}
    |S_m-S_n| &= |\sum_{k=0}^{m} r^{k} - \sum_{k=0}^{n} r^{k}| \\
    &= |\sum_{k=n+1}^{m} r^{k}| \\
    &= |r^{n+1} + r^{n+2} + \cdots + r^{m}| \\
    &\leq |r|^{n+1} + \cdots + |r|^{m} \\
    &= \frac{|r|^{n+1}(1-|r|^{m-n}) }{1-|r|}\\
    &= \frac{|r|^{n+1}-|r|^{m+1}}{1-|r|} \\
    &\leq \frac{|r|^{n}-|r|^{m}}{1-|r|}
\end{align}
So we want to set $\frac{|r|^{n}}{1-|r|} < \frac{\epsilon}{2}$ which means we want $N > \log_{|r|}{\frac{\epsilon \cdot (1-|r|)}{2}}$.

Then we proceed with the proof as standard (omitted here since I'm lazy \ldots).

4.11 (a) $70+7 \sum_{k=0}^{\infty}\frac{1}{10}^k$

(b) $70 \sum_{k=0}^{\infty} \frac{1}{10}^k$

We get the answer $\frac{700}{9}$ from both (a) and (b).

(c) If a number $q$ has a repeating decimal $m$ every $n$ digits, then represent
the decimal as $a+\sum_{k=1}^{\infty}m{(\frac{1}{10^n})}^k=\frac{m-am+a10^n}{10^n-m}$ which means $q$ is rational.

4.12 (a) If $\sum a_k$ converges absolutely, meaning $\sum |a_k|$ converges, since $b_k$ is a subsequence of $a_k$, the number of terms of $b_k$ is at most that of $a_k$. Thus,
it follows that \[\sum |b_k| \leq \sum |a_k|\] and so by comparison test, since $\sum |a_k|$ converges, $\sum |b_k|$ converges, and the result follows from Proposition 4.18.

(b) Consider the example from Exercise 4.5. That is, consider $(a_k)={(-1)}^{k+1}\frac{1}{n}$. This converges 
conditionally, and if we pick the subsequence $(a_{2k})$, it diverges to $-\infty$.